\documentclass[11pt,a4paper]{article}
\usepackage[top=1cm, bottom=1cm, left=2cm, right=2cm]{geometry}
\usepackage{fontspec}
\setmainfont{Liberation Serif}

\title{\bf{Python Workshop Exercises}}
\author{Baiju Muthukadan \\ ZeOmega, Bangalore \\ FOSSMeet'14, NIT Calicut}
\date{Feb 15, 2014}

\begin{document}
\maketitle

\vspace*{2cm}

\centerline{\LARGE\bf Exercises}
\section*{Exercise 1}
Write a Python program to print "Hello, World!" and save this in a
file named {\it helloworld.py}.  Make this program executable and run
it like: {\it./helloworld.py}

\section*{Exercise 2}
Write a Python program ({\it ex2.py}) to swap values of two variables.

\section*{Exercise 3}
Write a program that asks for two numbers.  If the sum of the numbers
is greater than 100, print "That is a big number."

\section*{Exercise 4}
Write a program that asks the user their name, if they enter your name
say "That is a nice name", if they enter "John Cleese" or "Michael
Palin", tell them how you feel about them ;), otherwise tell them "You
have a nice name."

\section*{Exercise 5}
Rewrite the below program ({\it ex5.py}) to have a separate function
for the area of a square, the area of a rectangle, and the area of a
circle (3.14 * radius ** 2). This program should include a menu
interface.

\begin{verbatim}
# By Amos Satterlee
print
def hello():
    print 'Hello!'
 
def area(width, height):
    return width * height
 
def print_welcome(name):
    print 'Welcome,', name
 
name = raw_input('Your Name: ')
hello(),
print_welcome(name)
print
print 'To find the area of a rectangle,'
print 'enter the width and height below.'
print
w = input('Width: ')
while w <= 0:
    print 'Must be a positive number'
    w = input('Width: ')
 
h = input('Height: ')
while h <= 0:
    print 'Must be a positive number'
    h = input('Height: ')
 
print 'Width =', w, 'Height =', h, 'so Area =', area(w, h)
\end{verbatim}

\section*{Exercise 6}
Expand the {\it ex6.py} program given below so it has a menu giving
the option of taking the test, viewing the list of questions and
answers, and an option to quit. Also, add a new question to ask, "What
noise does a truly advanced machine make?" with the answer of "ping".

\begin{verbatim}
## This program runs a test of knowledge
 
# First get the test questions
# Later this will be modified to use file io.
def get_questions():
    # notice how the data is stored as a list of lists
    return [["What color is the daytime sky on a clear day? ", "blue"],
            ["What is the answer to life, the universe and everything? ", "42"],
            ["What is a three letter word for mouse trap? ", "cat"]]
 
# This will test a single question
# it takes a single question in
# it returns True if the user typed the correct answer, otherwise False
 
def check_question(question_and_answer):
    # extract the question and the answer from the list
    question = question_and_answer[0]
    answer = question_and_answer[1]
    # give the question to the user
    given_answer = raw_input(question)
    # compare the user's answer to the testers answer
    if answer == given_answer:
        print "Correct"
        return True
    else:
        print "Incorrect, correct was:", answer
        return False
 
# This will run through all the questions
def run_test(questions):
    if len(questions) == 0:
        print "No questions were given."
        # the return exits the function
        return
    index = 0
    right = 0
    while index < len(questions):
        # Check the question
        if check_question(questions[index]):
            right = right + 1
            index = index + 1
        # go to the next question
        else:
            index = index + 1
    # notice the order of the computation, first multiply, then divide
    print "You got", right * 100 / len(questions),\
           "% right out of", len(questions)
 
# now let's run the questions
 
run_test(get_questions())
\end{verbatim}

\section*{Exercise 7}
Rewrite the below program ({\it ex7.py}) to use a random integer
between 0 and 99 instead of the hard-coded 78.  Use the Python
documentation to find an appropriate module and function to do this.

\begin{verbatim}
# Plays the guessing game higher or lower 
 
number = 78
guess = 0
 
while guess != number: 
    guess = input("Guess a number: ")
    if guess > number:
        print "Too high"
    elif guess < number:
        print "Too low"
 
print "Just right"
\end{verbatim}

\newpage

\centerline{\LARGE\bf Answers}
\section*{Answer 1}

\begin{enumerate}
\item Content of {\it helloworld.py}:
\begin{verbatim}
#!/usr/bin/env python

print "Hello, World!"
\end{verbatim}

\item Change mode from shell:
\begin{verbatim}
$ chmod +x helloworld.py
\end{verbatim}

\item Run program and verify output like this:
\begin{verbatim}
$ ./helloworld.py
Hello, World!
\end{verbatim}
\end{enumerate}

\section*{Answer 2}
\begin{enumerate}
\item Content of the file {\it ex2.py}:
\begin{verbatim}
x, y = 2, 3
x, y = y, x
print x, y
\end{verbatim}

\item Run program and verify output like this:
\begin{verbatim}
$ python ex2.py
3 2
\end{verbatim}

\end{enumerate}

\section*{Answer 3}
\begin{enumerate}
\item Content of the file {\it ex3.py}:
\begin{verbatim}
number1 = input('1st number: ')
number2 = input('2nd number: ')
if number1 + number2 > 100:
    print 'That is a big number.'
\end{verbatim}

\item Run program and verify output like this:
\begin{verbatim}
$ python ex3.py
1st number: 56
2nd number: 78
That is a big number.
\end{verbatim}

\end{enumerate}

\section*{Answer 4}
\begin{enumerate}
\item Content of the file {\it ex4.py}:
\begin{verbatim}
name = raw_input('Your name: ')
if name == 'Ada':
    print 'That is a nice name.'
elif name == 'John Cleese' or name == 'Michael Palin':
    print 'Wow. that\'s a great name!'
else:
    print 'You have a nice name.'
\end{verbatim}

\item Run program and verify output like this:
\begin{verbatim}
$ python ex4.py
Your name: Ada
That is a nice name.
$ python ex4.py
Your name: John Cleese
Wow. that's a great name!
$ python ex4.py
Your name: Jack
You have a nice name.
\end{verbatim}

\end{enumerate}

\section*{Answer 5}
\begin{enumerate}
\item Content of the file {\it ex5.py}:
\begin{verbatim}
def square(length):
    return length * length
 
def rectangle(width , height):
    return width * height
 
def circle(radius):
    return 3.14 * radius ** 2
 
def options():
    print
    print "Options:"
    print "s = calculate the area of a square."
    print "c = calculate the area of a circle."
    print "r = calculate the area of a rectangle."
    print "q = quit"
    print
 
print "This program will calculate the area of a square, circle or rectangle."
choice = "x"
options()
while choice != "q":
    choice = raw_input("Please enter your choice: ")
    if choice == "s":
        length = input("Length of square: ")
        print "The area of this square is", square(length)
        options()
    elif choice == "c":
        radius = input("Radius of the circle: ")
        print "The area of the circle is", circle(radius)
        options()
    elif choice == "r":
        width = input("Width of the rectangle: ")
        height = input("Height of the rectangle: ")
        print "The area of the rectangle is", rectangle(width, height)
        options()
    elif choice == "q":
        print "",
    else:
        print "Unrecognized option."
        options()
\end{verbatim}

\end{enumerate}

\section*{Answer 6}
\begin{enumerate}
\item Content of the file {\it ex6.py}:
\begin{verbatim}
## This program runs a test of knowledge
 
questions = [["What color is the daytime sky on a clear day? ", "blue"],
             ["What is the answer to life, the universe and everything? ", "42"],
             ["What is a three letter word for mouse trap? ", "cat"],
             ["What noise does a truly advanced machine make?", "ping"]]
 
# This will test a single question
# it takes a single question in
# it returns True if the user typed the correct answer, otherwise False
 
def check_question(question_and_answer):
    # extract the question and the answer from the list
    question = question_and_answer[0]
    answer = question_and_answer[1]
    # give the question to the user
    given_answer = raw_input(question)
    # compare the user's answer to the testers answer
    if answer == given_answer:
        print "Correct"
        return True
    else:
        print "Incorrect, correct was:", answer
        return False
 
# This will run through all the questions
 
def run_test(questions):
 
    if len(questions) == 0:
        print "No questions were given."
        # the return exits the function
        return
    index = 0
    right = 0
    while index < len(questions):
        # Check the question
        if check_question(questions[index]):
            right = right + 1
        # go to the next question
        index = index + 1
    # notice the order of the computation, first multiply, then divide
    print ("You got", right * 100 / len(questions),
           "% right out of", len(questions))
 
#showing a list of questions and answers
def showquestions(questions):
    q = 0
    while q < len(questions):
        a = 0
        print "Q:" , questions[q][a]
        a = 1
        print "A:" , questions[q][a]
        q = q + 1
 
# now let's define the menu function
def menu():
    print "-----------------"
    print "Menu:"
    print "1 - Take the test"
    print "2 - View a list of questions and answers"
    print "3 - View the menu"
    print "5 - Quit"
    print "-----------------"
 
choice = "3"
while choice != "5":
    if choice == "1":
        run_test(questions)
    elif choice == "2":
        showquestions(questions)
    elif choice == "3":
        menu()
    print
    choice = raw_input("Choose your option from the menu above: ")
\end{verbatim}
\end{enumerate}

\section*{Answer 7}
\begin{enumerate}
\item Content of the file {\it ex7.py}:
\begin{verbatim}
from random import randint
number = randint(0, 99)
guess = -1
while guess != number: 
    guess = input ("Guess a number: ")
    if guess > number:
        print "Too high"
    elif guess < number:
            print "Too low"
print "Just right"
\end{verbatim}
\end{enumerate}

\vspace*{2cm}
Please write your 2 minute feedback here: {\it http://bit.ly/fossmeet14feedback}

\end{document}