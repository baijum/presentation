\documentclass[12pt,handout]{beamer}
\hypersetup{pdfstartview={Fit}}
\usepackage[T1]{fontenc}
\usepackage{cmap}
\usepackage{hyperref}
\usepackage{listings}
\usetheme{default}
\usefonttheme{professionalfonts}
\usefonttheme{serif}
\usepackage{fontspec}
\setmainfont{Liberation Serif Bold}
\title[Python Workshop]{Python Workshop}
\author{Baiju Muthukadan \\ ZeOmega, Bangalore}
\institute{FOSSMeet'14, NIT Calicut}
\date{Feb 15, 2014}

\setbeamertemplate{frametitle}
  {\begin{centering}\smallskip
   \insertframetitle\par
   \smallskip\end{centering}}
\setbeamertemplate{itemize item}{$\bullet$}
\setbeamertemplate{navigation symbols}{}
\setbeamertemplate{footline}[text line]{%
    \hfill\strut{%
        \scriptsize\sf\color{black!60}%
        \quad\insertframenumber
    }%
    \hfill
}

% Define some colors:
\definecolor{DarkFern}{HTML}{407428}
\definecolor{DarkCharcoal}{HTML}{4D4944}
\colorlet{Fern}{DarkFern!85!white}
\colorlet{Charcoal}{DarkCharcoal!85!white}
\colorlet{LightCharcoal}{Charcoal!50!white}
\colorlet{AlertColor}{orange!80!black}
\colorlet{DarkRed}{red!70!black}
\colorlet{DarkBlue}{blue!70!black}
\colorlet{DarkGreen}{green!70!black}

% Use the colors:
%% \setbeamercolor{title}{fg=black}
%% \setbeamercolor{frametitle}{fg=black}
%% \setbeamercolor{normal text}{fg=black}
%% \setbeamercolor{block title}{fg=black,bg=Fern!25!white}
%% \setbeamercolor{block body}{fg=black,bg=Fern!25!white}
%% \setbeamercolor{alerted text}{fg=AlertColor}
%% \setbeamercolor{itemize item}{fg=Charcoal}

\setbeamercolor{alerted text}{fg=orange}
\setbeamercolor{background canvas}{bg=white}
\setbeamercolor{block body alerted}{bg=normal text.bg!90!black}
\setbeamercolor{block body}{bg=normal text.bg!90!black}
\setbeamercolor{block body example}{bg=normal text.bg!90!black}
\setbeamercolor{block title alerted}{use={normal text,alerted text},fg=alerted text.fg!75!normal text.fg,bg=normal text.bg!75!black}
\setbeamercolor{block title}{bg=blue}
\setbeamercolor{block title example}{use={normal text,example text},fg=example text.fg!75!normal text.fg,bg=normal text.bg!75!black}
\setbeamercolor{fine separation line}{}
\setbeamercolor{frametitle}{fg=red}
\setbeamercolor{item projected}{fg=black}
\setbeamercolor{normal text}{bg=white,fg=black}
\setbeamercolor{palette sidebar primary}{use=normal text,fg=normal text.fg}
\setbeamercolor{palette sidebar quaternary}{use=structure,fg=structure.fg}
\setbeamercolor{palette sidebar secondary}{use=structure,fg=structure.fg}
\setbeamercolor{palette sidebar tertiary}{use=normal text,fg=normal text.fg}
\setbeamercolor{section in sidebar}{fg=brown}
\setbeamercolor{section in sidebar shaded}{fg=grey}
\setbeamercolor{separation line}{}
\setbeamercolor{sidebar}{bg=red}
\setbeamercolor{sidebar}{parent=palette primary}
\setbeamercolor{structure}{bg=white, fg=black}
\setbeamercolor{subsection in sidebar}{fg=brown}
\setbeamercolor{subsection in sidebar shaded}{fg=grey}
\setbeamercolor{title}{fg=red}
\setbeamercolor{titlelike}{fg=white}

\setbeamertemplate{background canvas}{\includegraphics
	[width=\paperwidth,height=\paperheight]{fossmeet14.jpg}}

\newcommand{\code}[1]{\lstinline{#1}}

% \usepackage[printwatermark]{xwatermark}
% \usepackage{xcolor}
% \usepackage{graphicx}
% \usepackage{tikz}
% \usepackage{lipsum}

% \newsavebox\mybox
% \savebox\mybox{\tikz[color=red,opacity=0.3]\node{DRAFT};}
% \newwatermark*[
%   allpages,
%   angle=45,
%   scale=6,
%   xpos=-20,
%   ypos=15
% ]{\usebox\mybox}

\begin{document}

\begin{frame}
\titlepage
\end{frame}


\begin{frame}{Prerequisites}
\begin{itemize}
\item GNU/Linux
\item Python 2.7
\item Text editor
\end{itemize}
\end{frame}

\begin{frame}{Expectation from Participants}

\center \Large{Familiarity with programming in another language (C/C++/Java/C\#/Ruby/PHP)}

\end{frame}

\begin{frame}{About Me}

\begin{itemize}
\item Founded the SMC project in 2001 while studying at REC Calicut
\item Employed by FSF India in 2002-2003
\item Contributor to Zope project
\item Book Author: A Comprehensive Guide to Zope Component Architecture
\item During PyCON India 2013, received the first Kenneth Gonsalves Award
\end{itemize}

\end{frame}

\begin{frame}{Attribution}

This presentation and exercises are based on:
\\[.5cm]
\url{http://tdc-www.harvard.edu/Python.pdf}
\\
\url{http://en.wikibooks.org/wiki/Non-Programmer's_Tutorial_for_Python_2.6}

\end{frame}

\begin{frame}{Introduction}

\begin{itemize}
\item Free/Open source general-purpose language
\item Multi-paradigm -- Object Oriented, Procedural, Functional
\item Easy to interface with C/C++/ObjC/Java/Fortran
\item Great interactive environment
\item very clear, readable syntax
\item strong introspection capabilities
\item intuitive object orientation
\item full modularity, supporting hierarchical packages
\item exception-based error handling
\item very high level dynamic data types
\item extensive standard libraries and third party modules for virtually every task
\item embeddable within applications as a scripting interface

\end{itemize}
\end{frame}

\begin{frame}{Python Version}

\begin{itemize}
\item ``Current'' version is 3.3
\item ``Mainstream'' version is 2.7
\item Use 3.3 if dependencies are available, otherwise use 2.7

\end{itemize}
\end{frame}

\begin{frame}{Installation}

\begin{itemize}
\item Python comes pre-installed with GNU/Linux and Mac
\item Windows binaries from \url{http://python.org/}

\end{itemize}
\end{frame}

\begin{frame}[fragile]
\frametitle{Python Interactive Interpreter}

\begin{itemize}

\item{Interactive interface to Python

\small{
\begin{verbatim}
$ python
Python 2.7.5 (default, Nov 12 2013, 16:45:54)
[GCC 4.8.2 20131017 (Red Hat 4.8.2-1)] on linux2
Type "help", "copyright", "credits" or "license" for more information.
>>>
\end{verbatim}
}
}

\item{Python interpreter evaluates inputs

\small{
\begin{verbatim}
>>> 3 * (7 + 2)
27
>>> 'Hello ' + 'World!'
'Hello World!'
\end{verbatim}
}
}
\end{itemize}
\end{frame}

\begin{frame}[fragile]
\frametitle{Python Interactive Interpreter - Continued}

\begin{itemize}

\item{Python prompts with '>>>' (Primary) and '...' (Secondary)

\small{
\begin{verbatim}
>>> if 1 < 2:
...     print "1 is less than 2"
...
1 is less than 2
\end{verbatim}
}
}

\item To exit Python:  CTRL-D

\end{itemize}
\end{frame}

\begin{frame}[fragile]
\frametitle{Running Programs on GNU/Linux}

\begin{itemize}
\item Easy way to run a program:

\small{
\begin{verbatim}
$ python filename.py
\end{verbatim}
}

\item You could make the *.py file executable and add the
following \#!/usr/bin/env python to the top to make it
runnable.

\small{
\begin{verbatim}
$ ./filename.py
\end{verbatim}
}

\item Better cross platform solution is to make use setuptools console scripts
  (This is discussed later)
\end{itemize}

\end{frame}

\begin{frame}{Batteries Included}

Large collection of proven modules are included in the
standard distribution:
\\[.5cm]
\url{http://docs.python.org/2/library/index.html}
\\[.5cm]
And many more third part packages are available from PyPI
(Python Package Index Server aka. Cheeseshop):
\\[.5cm]
\url{https://pypi.python.org/pypi}

\end{frame}

\begin{frame}[fragile]
\frametitle{A Code Sample}

\small{
\begin{verbatim}
x = 34 - 23 # A comment.
y = "Hello" # Another one.
z = 3.45
if z == 3.45 or y == "Hello":
    x = x + 1
    y = y + " World" # String concat.
print x
print y
\end{verbatim}
}

\end{frame}

\begin{frame}{Enough to Understand the Code}

\begin{itemize}

\item Assignment uses = and comparison uses ==.
\item For numbers + - * / \% are as expected.
\begin{itemize}
\item Special use of + for string concatenation.
\item Special use of \% for string formatting (as with printf in C)
\end{itemize}
\item Logical operators are words (and, or, not)
not symbols
\item The basic printing command is print.
\item The first assignment to a variable creates it.
\begin{itemize}
\item Variable types don't need to be declared.
\item Python figures out the variable types on its own.
\end{itemize}

\end{itemize}

\end{frame}

\begin{frame}{Basic Datatypes}

\begin{itemize}

\item Integers (default for numbers)
  \code{z = 5 / 2} \# Answer is 2, integer division.
\item Floats
  x = 3.456
\item Strings
\item Can use \code{" "} or \code{' '} to specify.
  \code{"abc" 'abc'} (Same thing.)
\item Unmatched can occur within the string.
  \code{"matt's"}
\item Use triple double-quotes for multi-line strings or strings than contain both '
  and \code{"} inside of them:
  \code{"""a'b"c"""}

\end{itemize}
\end{frame}

\begin{frame}{Whitespace}

Whitespace is meaningful in Python: especially
indentation and placement of newlines.

\begin{itemize}
\item Use a newline to end a line of code.
\begin{itemize}
\item Use \ when must go to next line prematurely.
\item No braces { } to mark blocks of code in Python...
\end{itemize}
Use consistent indentation instead.
\begin{itemize}
\item The first line with less indentation is outside of the block.
\item The first line with more indentation starts a nested block
\end{itemize}
\item Often a colon appears at the start of a new block.
(E.g. for function and class definitions.)
\end{itemize}
\end{frame}

\begin{frame}[fragile]
\frametitle{Comments}

\begin{itemize}
\item Start comments with \# - the rest of line is ignored.
\item Can include a ''documentation string'' as the first line of any
new function or class that you define.
\item The development environment, debugger, and other tools use
it: it's good style to include one.
\small{
\begin{verbatim}
def my_function(x, y):
    """This is the docstring. This
    function does blah blah blah."""
    # The code would go here...
\end{verbatim}
}
\end{itemize}
\end{frame}

\begin{frame}[fragile]
\frametitle{Assignment}

\begin{itemize}
\item Binding a variable in Python means setting a name to hold a
reference to some object.
\begin{itemize}
\item Assignment creates references, not copies
\end{itemize}
\item Names in Python do not have an intrinsic type. Objects have
types.
\begin{itemize}
\item Python determines the type of the reference automatically based on the
data object assigned to it.
\end{itemize}
\item You create a name the first time it appears on the left side of
an assignment expression:
\small{
\begin{verbatim}
 x = 3
\end{verbatim}
}
\item A reference is deleted via garbage collection after any names
bound to it have passed out of scope.
\end{itemize}
\end{frame}

\begin{frame}[fragile]
\frametitle{Accessing Non-Existent Names}
\begin{itemize}
\item If you try to access a name before it's been properly created
(by placing it on the left side of an assignment), you'll get an
error.
\small{
\begin{verbatim}
>>> y
Traceback (most recent call last):
 File "<pyshell#16>", line 1, in -toplevel-
 y
NameError: name 'y' is not defined
>>> y = 3
>>> y
3
\end{verbatim}
}
\end{itemize}
\end{frame}

\begin{frame}[fragile]
\frametitle{Multiple Assignment}

\begin{itemize}
\item You can also assign to multiple names at the same time.
\small{
\begin{verbatim}
>>> x, y = 2, 3
>>> x
2
>>> y
3
\end{verbatim}
}
\end{itemize}
\end{frame}

\begin{frame}[fragile]
\frametitle{Naming Rules}

\begin{itemize}
\item Names are case sensitive and cannot start with a number.
They can contain letters, numbers, and underscores.\\
\small{
\begin{verbatim}
 bob Bob _bob _2_bob_ bob_2 BoB
\end{verbatim}
}
\item There are some reserved words:\\
\small{
\begin{verbatim}
 and, assert, break, class, continue, def, del, elif,
else, except, exec, finally, for, from, global, if,
import, in, is, lambda, not, or, pass, print, raise,
return, try, while
\end{verbatim}
}
\end{itemize}
\end{frame}

\begin{frame}[fragile]
\frametitle{User Input}

\begin{itemize}
\item Use {\it raw\_input} function to get user input as a string.
\item Use {\it input} function to get user input with evaluation
of the given expression
\small{
\begin{verbatim}
 name = raw_input("Enter name: ")
 age = raw_input("Enter age: ")
\end{verbatim}
}
\end{itemize}
\end{frame}

\begin{frame}[fragile]
\frametitle{Control of Flow - if conditions}
\small{
\begin{verbatim}
if x == 3:
    print "X equals 3."
elif x == 2:
    print "X equals 2."
else:
    print "X equals something else."

print "This is outside the 'if'."

\end{verbatim}
}
\end{frame}

\begin{frame}[fragile]
\frametitle{Control of Flow - while loop}
\small{
\begin{verbatim}
while x < 10:
    if x > 7:
        x += 2
        continue
    x = x + 1
    print "Still in the loop."
    if x == 8:
        break
print "Outside of the loop."
\end{verbatim}
}
\end{frame}

\begin{frame}[fragile]
\frametitle{Control of Flow - for loop}
\small{
\begin{verbatim}

for x in range(10):
    
    if x > 7:
        x += 2
        continue
    x = x + 1
    print "Still in the loop."
    if x == 8:
        break

print "Outside of the loop."
\end{verbatim}
}
\end{frame}


\begin{frame}[fragile]
\frametitle{Sequence Types}
\begin{enumerate}
\item Tuple
\begin{itemize}
\item A simple immutable ordered sequence of items
\item Items can be of mixed types, including collection types
\end{itemize}
\item Strings
\begin{itemize}
\item Immutable
\item Conceptually very much like a tuple
\end{itemize}
\item List
\begin{itemize}
\item Mutable ordered sequence of items of mixed types
\end{itemize}
\end{enumerate}
\end{frame}

\begin{frame}[fragile]
\frametitle{Similar Syntax}
\begin{itemize}
\item All three sequence types (tuples, strings, and lists)
share much of the same syntax and functionality.
\item Key difference:
\begin{itemize}
\item Tuples and strings are immutable
\item Lists are mutable
\end{itemize}
\item The operations shown in this section can be
applied to all sequence types
\begin{itemize}
\item most examples will just show the operation
performed on one
\end{itemize}
\end{itemize}
\end{frame}

\begin{frame}[fragile]
\frametitle{Sequence Types 1}
\begin{itemize}
\item Tuples are defined using parentheses (and commas).
\small{
\begin{verbatim}
>>> tu = (23, 'abc', 4.56, (2,3), 'def')
\end{verbatim}
}
\item Lists are defined using square brackets (and commas).
\small{
\begin{verbatim}
>>> li = ["abc", 34, 4.34, 23]
\end{verbatim}
}
\item Strings are defined using quotes (\code{", ', ''', """}).
\small{
\begin{verbatim}
>>> st = "Hello World"
>>> st = 'Hello World'
>>> st = '''This is a multi-line
string that uses triple single quotes.'''
>>> st = """This is a multi-line
string that uses triple double quotes."""
\end{verbatim}
}
\end{itemize}
\end{frame}


\begin{frame}[fragile]
\frametitle{Sequence Types 2}
\begin{itemize}
\item We can access individual members of a tuple, list, or string
  using square bracket ''array'' notation.
\item Note that all are 0 based...
\small{
\begin{verbatim}
>>> tu = (23, 'abc', 4.56, (2,3), 'def')
>>> tu[1] # Second item in the tuple.
 'abc'
>>> li = ["abc", 34, 4.34, 23]
>>> li[1] # Second item in the list.
 34
>>> st = "Hello World"
>>> st[1] # Second character in string.
 'e'
\end{verbatim}
}
\end{itemize}
\end{frame}


\begin{frame}[fragile]
\frametitle{Positive and negative indices}
\small{
\begin{verbatim}
>>> t = (23, 'abc', 4.56, (2,3), 'def')
\end{verbatim}
}
Positive index: count from the left, starting with 0.
\small{
\begin{verbatim}
 >>> t[1]
 'abc'
\end{verbatim}
}
Negative lookup: count from right, starting with -1.
\small{
\begin{verbatim}
 >>> t[-3]
 4.56
\end{verbatim}
}
\end{frame}


\begin{frame}[fragile]
\frametitle{Slicing: Return Copy of a Subset 1}
\small{
\begin{verbatim}
>>> t = (23, 'abc', 4.56, (2,3), 'def')
\end{verbatim}
}
Return a copy of the container with a subset of the original
members. Start copying at the first index, and stop copying
before the second index.
\small{
\begin{verbatim}
 >>> t[1:4]
 ('abc', 4.56, (2,3))
\end{verbatim}
}
You can also use negative indices when slicing.
\small{
\begin{verbatim}
 >>> t[1:-1]
 ('abc', 4.56, (2,3))
\end{verbatim}
}
\end{frame}


\begin{frame}[fragile]
\frametitle{Slicing: Return Copy of a Subset 2}
\small{
\begin{verbatim}
>>> t = (23, 'abc', 4.56, (2,3), 'def')
\end{verbatim}
}
Omit the first index to make a copy starting from the beginning
of the container.
\small{
\begin{verbatim}
 >>> t[:2]
 (23, 'abc')
\end{verbatim}
}
Omit the second index to make a copy starting at the first index
and going to the end of the container.
\small{
\begin{verbatim}
 >>> t[2:]
 (4.56, (2,3), 'def')
\end{verbatim}
}
\end{frame}


\begin{frame}[fragile]
\frametitle{Copying the Whole Sequence}
To make a copy of an entire sequence, you can use [:].
\small{
\begin{verbatim}
 >>> t[:]
 (23, 'abc', 4.56, (2,3), 'def')
\end{verbatim}
}
Note the difference between these two lines for mutable
sequences:
\small{
\begin{verbatim}
>>> list2 = list1 # 2 names refer to 1 ref
\end{verbatim}
}
\# Changing one affects both
\small{
\begin{verbatim}
>>> list2 = list1[:] # Two independent copies, two refs
\end{verbatim}
}
\end{frame}


\begin{frame}[fragile]
\frametitle{The 'in' Operator}
\begin{itemize}
\item Boolean test whether a value is inside a container:
\small{
\begin{verbatim}
>>> t = [1, 2, 4, 5]
>>> 3 in t
False
>>> 4 in t
True
>>> 4 not in t
False
\end{verbatim}
}
\item For strings, tests for substrings
\small{
\begin{verbatim}
>>> a = 'abcde'
>>> 'c' in a
True
>>> 'cd' in a
True
>>> 'ac' in a
False
\end{verbatim}
}
\item {\it in} keyword is used in the {\it for} loops
and list comprehensions.
\end{itemize}
\end{frame}


\begin{frame}[fragile]
\frametitle{The + Operator}
\begin{itemize}
\item The + operator produces a new tuple, list, or string whose
value is the concatenation of its arguments.
\small{
\begin{verbatim}
>>> (1, 2, 3) + (4, 5, 6)
 (1, 2, 3, 4, 5, 6)
>>> [1, 2, 3] + [4, 5, 6]
 [1, 2, 3, 4, 5, 6]
>>> "Hello" + " " + "World"
 'Hello World'
\end{verbatim}
}
\end{itemize}
\end{frame}


\begin{frame}[fragile]
\frametitle{The * Operator}
\begin{itemize}
\item The * operator produces a new tuple, list, or string that
''repeats'' the original content.
\small{
\begin{verbatim}
>>> (1, 2, 3) * 3
(1, 2, 3, 1, 2, 3, 1, 2, 3)
>>> [1, 2, 3] * 3
[1, 2, 3, 1, 2, 3, 1, 2, 3]
>>> "Hello" * 3
'HelloHelloHello'
\end{verbatim}
}
\end{itemize}
\end{frame}


\begin{frame}[fragile]
\frametitle{Tuples: Immutable}
\small{
\begin{verbatim}
>>> t = (23, 'abc', 4.56, (2,3), 'def')
>>> t[2] = 3.14
Traceback (most recent call last):
 File "<pyshell#75>", line 1, in -toplevel-
 tu[2] = 3.14
TypeError: object doesn't support item assignment
\end{verbatim}
}
You can't change a tuple.
You can make a fresh tuple and assign its reference to a previously used
name.
\small{
\begin{verbatim}
 >>> t = (23, 'abc', 3.14, (2,3), 'def')
\end{verbatim}
}
\end{frame}

\begin{frame}[fragile]
\frametitle{Lists: Mutable}
\small{
\begin{verbatim}
>>> li = ['abc', 23, 4.34, 23]
>>> li[1] = 45
>>> li
['abc', 45, 4.34, 23]
\end{verbatim}
}
\begin{itemize}
\item We can change lists in place.
\item Name {\it li} still points to the same memory reference when we're
done.
\item The mutability of lists means that they aren't as fast as tuples.
\end{itemize}
\end{frame}



\begin{frame}[fragile]
\frametitle{Operations on Lists Only 1}
\small{
\begin{verbatim}
>>> li = [1, 11, 3, 4, 5]
>>> li.append('a') # Our first exposure to method syntax
>>> li
[1, 11, 3, 4, 5, 'a']
>>> li.insert(2, 'i')
>>>li
[1, 11, 'i', 3, 4, 5, 'a']
\end{verbatim}
}
\end{frame}

\begin{frame}[fragile]
\frametitle{The extend method vs the + operator}
\begin{itemize}
\item + creates a fresh list (with a new memory reference)
\item extend operates on list li in place.

\small{
\begin{verbatim}
>>> li.extend([9, 8, 7])
>>>li
[1, 2, 'i', 3, 4, 5, 'a', 9, 8, 7]
\end{verbatim}
}

Confusing:
\item Extend takes a list as an argument.
\item Append takes a singleton as an argument.

\small{
\begin{verbatim}
>>> li.append([10, 11, 12])
>>> li
[1, 2, 'i', 3, 4, 5, 'a', 9, 8, 7, [10, 11, 12]]
\end{verbatim}
}

\end{itemize}
\end{frame}

\begin{frame}[fragile]
\frametitle{Operations on Lists Only 3}
\small{
\begin{verbatim}
>>> li = ['a', 'b', 'c', 'b']
>>> li.index('b') # index of first occurrence
1
>>> li.count('b') # number of occurrences
2
>>> li.remove('b') # remove first occurrence
>>> li
 ['a', 'c', 'b']
\end{verbatim}
}
\end{frame}


\begin{frame}[fragile]
\frametitle{Operations on Lists Only 4}
\small{
\begin{verbatim}
>>> li = [5, 2, 6, 8]
>>> li.reverse() # reverse the list *in place*
>>> li
 [8, 6, 2, 5]
>>> li.sort() # sort the list *in place*
>>> li
 [2, 5, 6, 8]
>>> li.sort(some_function)
 # sort in place using user-defined comparison
\end{verbatim}
}
\end{frame}

\begin{frame}[fragile]
\frametitle{Tuples vs. Lists}
\begin{itemize}
\item Lists slower but more powerful than tuples.
\item Lists can be modified, and they have lots of handy operations we can
perform on them.
\item Tuples are immutable and have fewer features.
\item To convert between tuples and lists use the list() and tuple()
functions:
\small{
\begin{verbatim}
li = list(tu)
tu = tuple(li)
\end{verbatim}
}
\end{itemize}
\end{frame}

\begin{frame}[fragile]
\frametitle{Dictionaries: A Mapping type}
\begin{itemize}
\item Dictionaries store a mapping between a set of keys
and a set of values.
\item Keys can be any immutable type.
\item Values can be any type
\item A single dictionary can store values of different types
\item You can define, modify, view, lookup, and delete
the key-value pairs in the dictionary.
\end{itemize}
\end{frame}


\begin{frame}[fragile]
\frametitle{Using dictionaries}
\small{
\begin{verbatim}
>>> d = {'user':'bozo', 'pswd':1234}
>>> d['user'] 
'bozo'
>>> d['pswd']
1234
>>> d['bozo']
Traceback (innermost last):
 File '<interactive input>' line 1, in ?
KeyError: bozo
>>> d = {'user':'bozo', 'pswd':1234}
>>> d['user'] = 'clown'
>>> d
{'user':'clown', 'pswd':1234}
>>> d['id'] = 45
>>> d
{'user':'clown', 'id':45, 'pswd':1234}
\end{verbatim}
}
\end{frame}

\begin{frame}[fragile]
\frametitle{Using dictionaries - Continued}
\small{
\begin{verbatim}
>>> d = {'user':'bozo', 'p':1234, 'i':34}
>>> del d['user'] # Remove one.
>>> d
{'p':1234, 'i':34}
>>> d.clear() # Remove all.
>>> d
{}
>>> d = {'user':'bozo', 'p':1234, 'i':34}
>>> d.keys() # List of keys.
['user', 'p', 'i']
>>> d.values() # List of values.
['bozo', 1234, 34]
>>> d.items() # List of item tuples.
[('user','bozo'), ('p',1234), ('i',34)]
\end{verbatim}
}
\end{frame}
 
\begin{frame}[fragile]
\frametitle{Functions}
\begin{itemize}
\item {\it def} creates a function and assigns it a name
\item return sends a result back to the caller
\item Arguments are passed by assignment
\item Arguments and return types are not declared
\small{
\begin{verbatim}
def <name>(arg1, arg2, ..., argN):
    <statements>
    return <value>

def times(x,y):
    return x*y
\end{verbatim}
}
\end{itemize}
\end{frame}

\begin{frame}[fragile]
\frametitle{Passing Arguments to Functions}
\begin{itemize}
\item Arguments are passed by assignment
\item Passed arguments are assigned to local names
\item Assignment to argument names don't affect the 
caller
\item Changing a mutable argument will affect the caller
  and it may not be the expected behaviour
\small{
\begin{verbatim}
def changer(x,y):
    x = 2 # changes local value of x only
    y[0] = 'hi' # changes shared object
\end{verbatim}
}

\end{itemize}
\end{frame}

\begin{frame}[fragile]
\frametitle{Optional Arguments}
\begin{itemize}

\item Can define defaults for arguments that need not be 
passed
\small{
\begin{verbatim}

def func(a, b, c=10, d=100):
    print a, b, c, d
>>> func(1,2)
1 2 10 100
>>> func(1,2,3,4)
1,2,3,4
\end{verbatim}
}
\end{itemize}
\end{frame}

\begin{frame}[fragile]
\frametitle{Gotchas}
\begin{itemize}
\item All functions in Python have a return value
\begin{itemize}
\item even if no return line inside the code.
\end{itemize}
\item Functions without a return return the special value 
{\it None}.
\item There is no function overloading in Python.
\begin{itemize}
\item Two different functions can't have the same name, even if they 
have different arguments.
\end{itemize}
\item Functions can be used as any other data type. 
They can be:
\begin{itemize}
\item Arguments to function
\item Return values of functions
\item Assigned to variables
\item Parts of tuples, lists, etc
\end{itemize}
\end{itemize}
\end{frame}

\begin{frame}[fragile]
\frametitle{Why Use Modules?}
\begin{itemize}
\item Code reuse
\begin{itemize}
\item Routines can be called multiple times within a program
\item Routines can be used from multiple programs
\end{itemize}
\item Namespace partitioning
\begin{itemize}
\item Group data together with functions used for that data
\end{itemize}
\item Implementing shared services or data
\begin{itemize}
\item Can provide global data structure that is accessed by multiple
  subprograms
\end{itemize}
\end{itemize}
\end{frame}

\begin{frame}[fragile]
\frametitle{Modules}
\begin{itemize}
\item Modules are functions and variables defined in 
separate files
\item Items are imported using {\it from} or {\it import}
\small{
\begin{verbatim}
    from module import function
    function()

    import module
    module.function()
\end{verbatim}
}
\item Modules are namespaces
\item Can be used to organize variable names, i.e.
\small{
\begin{verbatim}
 atom.position = atom.position - molecule.position
\end{verbatim}
}
\end{itemize}
\end{frame}

\begin{frame}[fragile]
\frametitle{String Formatting}
\begin{itemize}
\item Substitute values using a tuple
\small{
\begin{verbatim}
coins, amount, name = 2, 2.4, 'Tom'
out = '%s has %d coins worth a total of $%.02f' % (name,
                                          coins, amount)
print out

# Output: 'Tom has 2 coins worth a total of $2.40'
\end{verbatim}
}
\item Substitute value using a dictionary
\small{
\begin{verbatim}
data = {'coins': 2, 'amount': 2.4, 'name': 'Tom'}
out = '%(name)s has %(coins)d coins \
worth a total of $%(amount).02f' % data
print out

# Output: 'Tom has 2 coins worth a total of $2.40'
\end{verbatim}
}

\end{itemize}
\end{frame}


\begin{frame}[fragile]
\frametitle{Exceptions}
\small{
\begin{verbatim}
>>> try:
...     1 / 0
... except:
...     print('That was silly!')
... finally:
...     print('This gets executed no matter what')
... 
That was silly!
This gets executed no matter what
\end{verbatim}
}
\end{frame}

\begin{frame}[fragile]
\frametitle{File I/O}
\begin{itemize}
\item Reading file content:
\small{
\begin{verbatim}
fd = open('filename.txt', 'r')
for line in fd:
    print line
fd.close()

open('filename.txt').read()

open('filename.txt').readlines()
\end{verbatim}
}

\item Writing file content:

\small{
\begin{verbatim}
fd open('filename.txt', 'w')
fd.write('Hello, World!')
fd.write('\n')
fd.close()
\end{verbatim}
}
\end{itemize}
\end{frame}

\begin{frame}{What's next ?}

Documentation and pointers:

\begin{itemize}
\item \url{http://learnpythonthehardway.org/book/}
\item \url{http://www.reddit.com/r/LearnPython} (Ask your questions here)
\item \url{http://docs.python.org/2/}
\item \url{http://reddit.com/r/Python} (News)
\item \url{http://planet.python.org/} (Blog aggregator)
\item \url{http://www.pythonweekly.com/} (Newsletter)
\end{itemize}

\end{frame}

\begin{frame}{Thanks!}
\center \url{http://muthukadan.net}
\end{frame}


\end{document}
