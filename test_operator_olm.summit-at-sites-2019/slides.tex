\documentclass[aspectratio=169]{beamer}

\usepackage[utf8]{inputenc}
\usetheme{Madrid}
\usecolortheme{beaver}
\usepackage{fontspec}
\usepackage{listings}
\usepackage{fancyvrb}

\title{Test your Kubernetes operator with OLM}
\author{Baiju Muthukadan \and Avni Sharma}
\date{May 15, 2019}
\logo{\includegraphics[height=1.5cm]{images/Logo-RedHat-B-Color-RGB.png}}

\begin{document}
\beamertemplatenavigationsymbolsempty

\setmainfont
[ Path = ../fonts/,
UprightFont = DejaVuSerif.ttf,
ItalicFont = DejaVuSerif-Italic.ttf,
BoldFont = DejaVuSerif-Bold.ttf,
BoldItalicFont = DejaVuSerif-BoldItalic.ttf,
Numbers={Lining, Monospaced},
] {DejaVu Serif}

\setsansfont
[ Path = ../fonts/,
UprightFont = DejaVuSans.ttf,
ItalicFont = DejaVuSans-Oblique.ttf,
BoldFont = DejaVuSans-Bold.ttf,
BoldItalicFont = DejaVuSans-BoldOblique.ttf,
Numbers={Lining, Monospaced},
] {DejaVu Sans}

\setmonofont
[ Path = ../fonts/,
UprightFont = DejaVuSansMono.ttf,
ItalicFont = DejaVuSansMono-Oblique.ttf,
BoldFont = DejaVuSansMono-Bold.ttf,
BoldItalicFont = DejaVuSansMono-BoldOblique.ttf,
Numbers={Lining, Monospaced},
] {DejaVu Sans Mono}


\newfontfamily{\vollkorn}
[ Path = ../fonts/,
UprightFont = Vollkorn-Regular.otf,
ItalicFont = Vollkorn-Italic.otf,
BoldFont = Vollkorn-Bold.otf,
BoldItalicFont = Vollkorn-BoldItalic.otf,
Numbers={Lining, Monospaced},
] {Vollkorn}

\newfontfamily{\dejavuserif}
[ Path = ../fonts/,
UprightFont = DejaVuSerif.ttf,
ItalicFont = DejaVuSerif-Italic.ttf,
BoldFont = DejaVuSerif-Bold.ttf,
BoldItalicFont = DejaVuSerif-BoldItalic.ttf,
Numbers={Lining, Monospaced},
] {DejaVu Serif}

\newfontfamily{\dejavusans}
[ Path = ../fonts/,
UprightFont = DejaVuSans.ttf,
ItalicFont = DejaVuSans-Oblique.ttf,
BoldFont = DejaVuSans-Bold.ttf,
BoldItalicFont = DejaVuSans-BoldOblique.ttf,
Numbers={Lining, Monospaced},
] {DejaVu Sans}

\newfontfamily{\dejavumono}
[ Path = ../fonts/,
UprightFont = DejaVuSansMono.ttf,
ItalicFont = DejaVuSansMono-Oblique.ttf,
BoldFont = DejaVuSansMono-Bold.ttf,
BoldItalicFont = DejaVuSansMono-BoldOblique.ttf,
Numbers={Lining, Monospaced},
] {DejaVu Sans Mono}

\frame{\titlepage}

\begin{frame}
  \frametitle{About Us}

  \begin{itemize}
  \item<1-> Baiju is a Senior Software Engineer
  \item<2-> Avni is an Associate Software Engineer
  \item<3-> Both of us works on DevConsole Operator
  \end{itemize}

\end{frame}

\begin{frame}
  \frametitle{DevConsole Operator}

  \includegraphics[scale=.15]{images/Logo-Red_Hat-OpenShift_4-A-Standard-RGB.png}\\[.25in]
  
  Provides a developer-focused view in OpenShift 4

\end{frame}

\begin{frame}

  \includegraphics[scale=.25]{images/devconsole.png}
  
\end{frame}

\begin{frame}
  \frametitle{Operator Framework}

  \includegraphics[scale=.50]{images/operator_logo_framework_color.png}
  
  A toolkit to manage Kubernetes native applications, called Operators.

\end{frame}

\begin{frame}
  \frametitle{Operator Framework Parts}

  \includegraphics[scale=.50]{images/operator_logo_sdk_color.png}

  \includegraphics[scale=.50]{images/operator_logo_lifecycle_manager_color.png}

  \includegraphics[scale=.50]{images/operator_logo_metering_color.png}

\end{frame}

\begin{frame}
  \frametitle{Operator Lifecycle Manager (OLM)}

  Oversees installation, updates, and management of the lifecycle of
  all of the Operators (and their associated services) running across
  a Kubernetes cluster.

\end{frame}

\begin{frame}

  \frametitle{Two resources}

CatalogSource

a repository of CSVs, CRDs, and packages that define an application

Subscription

used to keep CSVs up to date by tracking a channel in a package

\end{frame}

\begin{frame}[fragile]
  \frametitle{Makefile}

  \begin{Verbatim}[fontsize=\small]
.PHONY: test-e2e-olm-ci
test-e2e-olm-ci: ./vendor
  $(Q)sed -e "s,REPLACE_IMAGE,registry.svc.ci.openshift.org/
    ${OPENSHIFT_BUILD_NAMESPACE}/stable:devconsole-operator-registry,"
    ./test/e2e/catalog_source_OS4.yaml | oc apply -f -
  $(Q)oc apply -f ./test/e2e/subscription_OS4.yaml
  $(eval DEPLOYED_NAMESPACE := openshift-operators)
  $(Q)./hack/check-crds.sh
  $(Q)operator-sdk test local ./test/e2e --no-setup
    --go-test-flags "-v -timeout=15m"

./vendor: Gopkg.toml Gopkg.lock
  $(Q)dep ensure ${V_FLAG} -vendor-only
  \end{Verbatim}
\end{frame}

\begin{frame}[fragile]
  \frametitle{CatalogSource}

  \begin{Verbatim}[fontsize=\small]
# Ref. https://github.com/operator-framework/operator-lifecycle-manager
apiVersion: operators.coreos.com/v1alpha1
kind: CatalogSource
metadata:
  name: my-catalog
  namespace: openshift-operator-lifecycle-manager
spec:
  sourceType: grpc
  image: REPLACE_IMAGE
  displayName: Community Operators
  publisher: Red Hat
  \end{Verbatim}
\end{frame}

\begin{frame}[fragile]
  \frametitle{Subscription}

  \begin{Verbatim}[fontsize=\small]
# Ref. https://github.com/operator-framework/operator-lifecycle-manager
apiVersion: operators.coreos.com/v1alpha1
kind: Subscription
metadata:
  name: my-devconsole
  namespace: openshift-operators
spec:
  channel: alpha
  name: devconsole
  source: my-catalog
  sourceNamespace: openshift-operator-lifecycle-manager
  \end{Verbatim}
\end{frame}

\begin{frame}[fragile]
  \frametitle{OpenShift CI Configuration}

  \begin{Verbatim}[fontsize=\small]
base_images:
  operator-registry:
    name: "4.0"
    namespace: ocp
    tag: operator-registry
images:
- from: operator-registry
  dockerfile_path: openshift-ci/Dockerfile.registry.intermediate
  to: operator-registry-base
- from: operator-registry-base
  dockerfile_path: openshift-ci/Dockerfile.registry.build
  to: devconsole-operator-registry
tests:
- as: e2e-olm-ci
  commands: make test-e2e-olm-ci
  openshift_installer_src:
    cluster_profile: aws
  \end{Verbatim}
\end{frame}

\begin{frame}[fragile]
  \frametitle{File: openshift-ci/Dockerfile.registry.intermediate}

  \begin{Verbatim}[fontsize=\small]
FROM quay.io/openshift/origin-operator-registry:latest
  \end{Verbatim}
\end{frame}

\begin{frame}[fragile]
  \frametitle{File: openshift-ci/Dockerfile.registry.build}

  \begin{Verbatim}[fontsize=\small]
FROM quay.io/openshift/origin-operator-registry:latest

ARG version=0.1.0

COPY manifests manifests
COPY deploy/crds/*.yaml manifests/devconsole/${version}/

RUN sed -e "s,REPLACE_IMAGE,registry.svc.ci.openshift.org/
  ${OPENSHIFT_BUILD_NAMESPACE}/stable:devconsole-operator,"
  -i manifests/devconsole/${version}/devconsole-operator.v${version}
  .clusterserviceversion.yaml
RUN initializer

USER 1001
EXPOSE 50051
CMD ["registry-server", "--termination-log=log.txt"]
  \end{Verbatim}
\end{frame}

\begin{frame}[fragile]

  \frametitle{Directory: manifests/}
  
  \begin{Verbatim}[fontsize=\small]
.
└── devconsole
    ├── 0.1.0
    │   └── devconsole-operator.v0.1.0.clusterserviceversion.yaml
    └── devconsole.package.yaml
  \end{Verbatim}
\end{frame}


\begin{frame}[fragile]

  \frametitle{File: manifests/devconsole/devconsole.package.yaml}
  
  \begin{Verbatim}[fontsize=\small]
packageName: devconsole
channels:
- name: alpha
  currentCSV: devconsole-operator.v0.1.0
  \end{Verbatim}
\end{frame}

\begin{frame}[fragile]

  \frametitle{File: manifests/devconsole/0.1.0/devconsole-operator.v0.1.0.clusterserviceversion.yaml}
  
  \begin{Verbatim}[fontsize=\small]
apiVersion: operators.coreos.com/v1alpha1
kind: ClusterServiceVersion
metadata:
  annotations:
    capabilities: Full Lifecycle
    description: The operator that enables a developer-focused perspective in OpenShift 4
    categories: "Developer Tools"
  name: devconsole-operator.v0.1.0
  namespace: placeholder
spec:
  apiservicedefinitions: {}
  customresourcedefinitions:

  ...
  \end{Verbatim}
\end{frame}

\begin{frame}
  \begin{center}
    {\huge Thank You!}\\[1cm]
    \url{https://github.com/redhat-developer/devconsole-operator}
  \end{center}
\end{frame}

\end{document}
